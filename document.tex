\documentclass{article}
\usepackage{xcolor} % for defining colors
\usepackage{geometry} % to ajust margins
\usepackage{setspace} % to use \setlength and \setstretch
\usepackage{anyfontsize} % to ajust font sizes with \fontsize
\usepackage{amssymb} % to use \veebar (xor)

% Font definition
\usepackage[T1]{fontenc}
\usepackage{mathpazo}
\usepackage{titlesec}

% Title formating
\titleformat*{\section}{\LARGE\bfseries}
\titleformat*{\subsection}{\Large\bfseries}

% Adjusting margins
\geometry{a4paper, left=30mm, right=30mm, top=30mm, bottom=30mm} 
% \setlength{\parindent}{0pt} % This line removes indentation
% \setstretch{1.15} % This line changes the line spacing
\setlength{\parskip}{1em} % This line adds 0.5 em between paragraphs

\definecolor{pagecolor}{RGB}{40, 40, 40} % Dark gray color
\pagecolor{pagecolor} % background color
\color{white} % font color
\sloppy % for full hbox errors

\begin{document}
  \fontsize{14}{15}\selectfont % font size 14, line spacing as it was 15
	% Title and Author Information
	\title{Discrete Math - Exercises}
	\author{André Carvalho}
	% \date{17/05/2023} % Use \date{specific date} to set a specific date
	
	\maketitle
	
	\setcounter{section}{-1} % start to count from zero
	
	\section{Chapter 0}
	Introduction and Preliminares
	
	\subsection{Investigate, pg. 2}
	1. Every person will shake hands with 9 other people. And a handshake is something that "belongs" to both participants. So it's $\frac{10 * 9}{2} = 45$.
	  
	2. Zeno ate! $2 * 26 = 52$ hotdogs. The total hotdogs eaten was 
	$$
	\sum_{i=1}^{26} i*2 = 2 * \sum_{i=1}^{26} i = 2 * \frac{26 * 27}{2} = 351
	$$

	3. Let's break the propositions:

	p: If this chess is empty

	q: The other chest's message is true

	The left chest message is $p \rightarrow q$.

	r: This chest is filled with treasure
	s: The other chest contains deadly scorpions

	The right chest message is $p \wedge q$.

	If I open left chest and it's empty, so, as $p \rightarrow q$:

	$$T \rightarrow q$$

	For this to be true, q must be true. But if q is true, them the right chest message is true, otherwise left chest message must be false, what is a paradox. So the first chest could not be empty and therefore the left chess message could never be the true one.

	For the right chess message to be true, either this chess is filled with treasure of the other is filled with deadly scorpions. I cannot guarantee that this chest is filled with treasure, but is at least safe to open and test it.
	
	4. No, it is not. There's no physical configuration that make possible to draw connections for all other towns without intersections.

	\subsection{Investigate, pg. 4}

  Let's start by analyzing each of the troll's statements:

  Troll 1: If I am a knave, then there are exactly two knights here.

  Troll 2: Troll 1 is lying.

  Troll 3: Either we are all knaves or at least one of us is a knight.
  
  First, we need to note that there cannot be three knaves because Troll 3's statement would be a lie in this scenario, which can't be the case.

  Second, there cannot be three knights, as that would make Troll 1's statement a lie, which again can't be the case.

  Given these two observations, we know there must be at least one knight and at least one knave. This leaves us with three possibilities:

  There are two knights and one knave.

  There are two knaves and one knight.

  Let's consider these possibilities.

  If there are two knights and one knave, the only way this can happen given the trolls' statements is if Troll 1 is a knight, Troll 2 is a knave, and Troll 3 is a knight. This is because, if Troll 1 were a knave, then his statement would be true, which can't be the case.

  If there are two knaves and one knight, the only way this can happen is if Troll 1 is a knave, Troll 2 is a knave, and Troll 3 is a knight. This is because, if Troll 3 were a knave, then his statement would be true, which can't be the case.

  So, there's a contradiction in your original reasoning: you've got two scenarios, but one of them must be wrong because in each scenario, we have a different number of knights and knaves.

  After a detailed analysis of the trolls' statements, we can conclude that the only possible configuration is that Troll 1 is a knight, Troll 2 is a knave, and Troll 3 is a knight.

  \textbf{Truth Tables:}

  Troll 1: If I am a knave, then there are exactly two knights here.

  This statement can be translated as:

  p: I am a knave (Troll 1)

  q: There are exactly two knights here

  {
  \setlength{\tabcolsep}{10pt} % increase space between columns
  \renewcommand{\arraystretch}{1.5} % increase space between rows
  \begin{tabular}{|c|c|c|}
    \hline
    $p$ (Troll 1) & $q$ (Two knights) & $p \rightarrow q$ \\
    \hline
    T (knave) & F (not two knights) & F \\
    T (knave) & T (two knights) & T \\
    F (knight) & F (not two knights) & T \\
    F (knight) & T (two knights) & T \\
    \hline
  \end{tabular}
  }

  Troll 2: Troll 1 is lying.

  This statement can be translated as:

  p: Troll 1 is not telling the truth

  {
  \setlength{\tabcolsep}{10pt} % increase space between columns
  \renewcommand{\arraystretch}{1.5} % increase space between rows
  \begin{tabular}{|c|c|}
    \hline
    $p$ (Troll 1) & $\neg p$ (Troll 2) \\
    \hline
    T (truth) & F (lie) \\
    F (lie) & T (truth) \\
    \hline
  \end{tabular}
  }

  Troll 3: Either we are all knaves or at least one of us is a knight.

  This statement can be translated as:

  p: We are all knaves (Troll 3)

  q: At least one of us is a knight

  {
  \setlength{\tabcolsep}{10pt} % increase space between columns
  \renewcommand{\arraystretch}{1.5} % increase space between rows
  \begin{tabular}{|c|c|c|}
    \hline
    $p$ (All knaves) & $q$ (One knight) & $p \vee q$ (Troll 3) \\
    \hline
    T (all knaves) & F (no knights) & T \\
    T (all knaves) & T (one knight) & T \\
    F (not all knaves) & F (no knights) & F \\
    F (not all knaves) & T (one knight) & T \\
    \hline
  \end{tabular}
  }

  The configuration that makes all statements correct (given that a knight always tells the truth and a knave always lies) is the following:

  Troll 1 is a knight (because "If I am a knave, then there are exactly two knights here" is true when Troll 1 is a knight and there are not exactly two knights)

  Troll 2 is a knave (because "Troll 1 is lying" is a lie when Troll 1 is a knight)

  Troll 3 is a knight (because "Either we are all knaves or at least one of us is a knight" is true when there's at least one knight)

  Therefore, the correct answer is: Troll 1 is a knight, Troll 2 is a knave, and Troll 3 is a knight.

	\subsection{Investigate, pg. 15}

  Statements 2 and 4 are equivalent. 1 implies 3. 3 implies no one, and 4 implies no one. 

	\subsection{Exercises, pg. 17}

  1. a) Atomic statement;

  b) Atomic statement;

  c) Molecular statement.

  2. a) Not a statement;
  
  b) Atomic statement

  c) Molecular statement, disjunction;

  d) Molecular statement, conditional;

  e) Molecular statement, disjunction;

  f) Atomic statement

  3. a) $P \wedge Q$

  b) $P \rightarrow \neg Q$

  c) Jack passed math or Jill passed math

  d) If is not the case that Jack or Jill passed math, then Jill passed math

  e) i) Nothing. ii) Jack did not pass math.

  4. a) Impossible to determine;
  
  b) A true statement, regardless 13 is your favorite number;

  c) A true statement, regardless 13 is your favorite number;

  d) A true statement regardless;

  e) Impossible to determine, as I don't know if 13 is your favorite number;

  f) False, regardless;

  g) True, regardless.

  5. p: "The square is blue";

  q: "The triangle is green";
	
  {
  \setlength{\tabcolsep}{10pt} % increase space between columns
  \renewcommand{\arraystretch}{1.5} % increase space between rows
  \begin{tabular}{|c|c|c|}
    \hline
    $p$ & $q$ & $p \rightarrow q$ \\
    \hline
    T & T & T \\
    T & F & F \\
    F & T & T \\
    F & F & T \\
    \hline
  \end{tabular}
  }

  If $p \rightarrow q$ is true, then $q$ could not be false.

  a) False;

  b) Impossible to determine;

  c) True, as the triangle is green;

  d) Impossible to determine;

  e) True, as the triangle is green.

  6. p: "The square is blue";

  q: "The triangle is green";

  If the converse is false, the only possible solution is that $p$ is false, and $q$ is true. In other words, the triangle is green, and the square is not blue.
	
  {
  \setlength{\tabcolsep}{10pt} % increase space between columns
  \renewcommand{\arraystretch}{1.5} % increase space between rows
  \begin{tabular}{|c|c|c|c|}
    \hline
    $p$ & $q$ & $p \rightarrow q$ & $q \rightarrow p$ \\
    \hline
    F & T & T & F \\
    \hline
  \end{tabular}
  }

  a) False;

  b) True;

  c) False;

  d) True.

  7. a) Neither;

  b) Contrapositive;

  c) Converse;

  d) Neither;

  e) Neither;

  f) Neither;

  8. a) If Oscar drinks milk, them he eats Chinese food;

  b) If Oscar does not drink milk, them he does not eat Chinese food;

  c) Yes, it's possible. If the contrapositive in that case is false, them I have to conclude that Oscar indeed does not eat Chinese food;

  d) No, I can't. It still possible that Oscar eats or not Chinese food and the statement be true whatsoever.

  e) This is not a possible situation.

  9. a) Not equivalent and not converse;

  b) This is the converse;

  c) This is the converse also;

  d) A equivalent what to express the original statement;

  10. a) If you exercise, then you lose weight;

  b) If you exercise, then you lose weight;

  c) If you're American, then you're patriotic;

  d) If you're American, then you're patriotic;

  e) If a number is a rational number, then it is a real number;

  f) If a number is not even, then it is prime;

  g) If the Broncos play the Super Bowl, then they win the Super Bowl;

  11. 


  a) $p$: You win the lottery; $q$: You will be rich.

  Original proposition: $p \rightarrow q$. Let's call it "1" from now on. 

  Proposition stated on item a: $p \veebar \neg q$. Let's call it "2".

  Truth table of "1".

  {
  \setlength{\tabcolsep}{10pt} % increase space between columns
  \renewcommand{\arraystretch}{1.5} % increase space between rows
  \begin{tabular}{|c|c|c|}
    \hline
    $p$ & $q$ & $p \rightarrow q$ \\
    \hline
    T & T & T \\
    T & F & F \\
    F & T & T \\
    F & F & T \\
    \hline
  \end{tabular}
  }
  
  Truth table of "2".

  {
  \setlength{\tabcolsep}{10pt} % increase space between columns
  \renewcommand{\arraystretch}{1.5} % increase space between rows
  \begin{tabular}{|c|c|c|}
    \hline
    $p$ & $q$ & $p \veebar \neg q$ \\
    \hline
    T & T & T \\
    T & F & F \\
    F & T & F \\
    F & F & T \\
    \hline
  \end{tabular}
  }

  These two truth tables are not equivalent.


  This is not equivalent to the original statement nor it is its converse;

  b) This is not equivalent to the original statement nor it is its converse;

  c) This is not equivalent to the original statement nor it is its converse;

  d) This is equivalent to the original statement;

  e) This is not equivalent to the original statement nor it is its converse;

  f) This is the converse of the original statement;

  g) This is equivalent to the original statement;

  h)  This is equivalent to the original statement;

  i) This is not equivalent to the original statement nor it is its converse;

  j) This is not equivalent to the original statement nor it is its converse;

  k) This is the contrapositive;

  l) This is not equivalent to the original statement nor it is its converse;


\end{document}
