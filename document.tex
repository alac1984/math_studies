\documentclass{article}
\usepackage{xcolor}
\usepackage{geometry}
\usepackage{setspace}

\geometry{a4paper, left=20mm, right=20mm, top=20mm, bottom=20mm}
\setlength{\parindent}{0pt} % This line removes indentation
\setlength{\parskip}{1em} % This line adds 1 em between paragraphs
\setstretch{1.15} % This line changes the line spacing

\definecolor{pagecolor}{RGB}{40, 40, 40} % Dark gray color
\pagecolor{pagecolor}
\color{white}

\begin{document}

	
	% Title and Author Information
	\title{Discrete Math - Exercises}
	\author{André Carvalho}
	\date{17/05/2023} % Use \date{specific date} to set a specific date
	
	\maketitle
	
	\setcounter{section}{-1}
	
	% Introduction
	\section{Chapter 0}
	Introduction and Preliminares
	
	\subsection{Investigate, pg. 2}
	1. Every person will shake hands with 9 other people. And a handshake is something that "belongs" to both participants. So it's $\frac{10 * 9}{2} = 45$.
	  
	2. Zeno ate! $2 * 26 = 52$ hotdogs. The total hotdogs eaten was 
	$$
	\sum_{i=1}^{26} i*2 = 2 * \sum_{i=1}^{26} i = 2 * \frac{26 * 27}{2} = 351
	$$

	3. Let's break the propositions:

	p: If this chess is empty

	q: The other chest's message is true

	The left chess message is $p \rightarrow q$.

	r: This chest is filled with treasure
	s: The other chest contains deadly scorpions

	The right chess message is $p \wedge q$.

	If I open left chest and it's empty, so, as $p \rightarrow q$:

	$$T \rightarrow q$$

	For this to be true, q must be true. But if q is true, them the right chest message is true, otherwise left chest message must be false, what is a paradox. So the first chest could not be empty and therefore the left chess message could never be the true one.

	For the right chess message to be true, either this chess is filled with treasure of the other is filled with deadly scorpions. I cannot guarantee that this chest is filled with treasure, but is at least safe to open and test it.
	
	4. No, it is not. There's no physical configuration that make possible to draw connections for all other towns without intersections.

	\subsection{Investigate, pg. 4}

	Troll 1: $p \rightarrow q$

	Troll 2: $\sim(p \rightarrow q)$

	Troll 3: $r \wedge s$

	Assuming Troll 1 is a knight, them Troll 2 is a knave and Troll 3 is a knight also. And them Troll 1 is wrong, because there's no 2 knights, and we're in a paradox. This is not a valid scenario.

	Assuming Troll 2 is a knight, them Troll 1 is a knave. If Troll 1 is a knave ('p'), as its sentence is a implication, the only way he can be lying is with proposition 'q' being false, so there could not be two knights. So Troll 3 cannot be right, but it is, and we're in another paradox here. Another invalid scenario.

	Assuming Troll 3 is a knight, so Troll 1 is a knave


	
	
\end{document}
